\section{Macros}

Macros are a tool to make the compiler write code for us.

\subsection{Declarative Macros}

Declarative Macros are defined using \textit{macro\_rules!} and can be used by calling \textit{macro\_name!}.
They are useful to generate code that have the same structure and is written multiple times like tests or implementations.

The syntax is similar to a match statement, where for each matcher we have to indicate how to transcribe it into code.
The syntax of the matcher is a bit tricky but essentially consists of identifiers and expressions encapsulated by \textit{\$()}.

\subsection{Procedural Macros}

Procedural Macros generate code from some given inputs, there are 3 types:
\begin{itemize}
    \item Function-like macros, using \textit{macro\_rules!},
    \item Attribute macros, like \textit{\#[test]},
    \item Derive macros, like \textit{\#[derive(Debug)]}.
\end{itemize}

The \textbf{Derive} macros are used only to automate implementation of traits when the automation is possible.

The \textbf{Function-like} macros are essentially the same as declarative macros except that the compiler doesn't check for correctness.
So we can use them in the rare case that the matching expressions get too complicated and want to work around them.

The \textbf{Attribute} macros are mainly used in test generation, framework annotations like \textit{\#[tokio::main]} and other complicated parts of the code.
