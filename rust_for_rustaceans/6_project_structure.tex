\section{Project Structure}

\subsection{Features}

Features are flags that Cargo uses to add optional functionality to your project and its dependencies.

In a project some part of the code can be left as optional behind a feature in order to make the compilation and de dependencies easier for the user.
This can be done by:

\begin{lstlisting}[language=Rust, style=boxed]
    #[cfg(feature = "some-feature")]
\end{lstlisting}

\subsection{Work-spaces}

The project can be split in different parts in a workspace to reduce the compile times and the usability,
so one change is made only the modified part of the project will be compiled.
To do it add this to the \textit{Cargo.toml} file:

\begin{lstlisting}[language=Rust, style=boxed]
    [workspace]
    members = [
    	"foo",
    	"bar/one",
    	"bar/two",
    ]   
\end{lstlisting}


\subsection{Project Configuration}

On the \textit{Cargo.toml} file we can add some metadata about the author, path to the project or to the documentation, etc.

Also the build configuration can be specified from here so the compiler knows what to do for every kind of build that encounters.
We can specify the optimization level, the debug options or how the program should \textbf{panic}.
We can even set different optimization options to the different dependencies.

\subsection{Conditional Compilation}

We can specify which parts of the project will be compiled depending on some options like, features, operating systems, context (debug, test…) or cpu architecture.
This conditioning can also be applied to dependencies.

\subsection{Versioning}

There are three levels of versioning,

\begin{enumerate}
    \item Breaking changes, that require a major version change
    \item Additions, that require a minor version change
    \item Bug fix, that require a patch version change
\end{enumerate}

On top of that if the project is public we should respect the \textbf{MSRV} which stands for \textit{Minimum Supported Rust Version},
for our project and should be supported for at least 6  to 12 month.

To not make the version number too large we can name with a suffix like \textit{-alpha.1} the unreleased versions.