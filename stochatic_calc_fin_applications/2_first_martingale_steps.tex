\section{First Martingale Steps}

A sequence of random variables ${M_n}$ is a martingale with respect to ${X_n}$ an other sequence of random variables,
if exists $f_n: \mathbb{R}^n \rightarrow \mathbb{R}$ such that $M_n = f_n(X_1, \dots, X_n)$, and
\begin{equation*}
    \mathbb{E}[M_n | X_1, \dots, X_{n - 1}] = M_{n - 1}.
\end{equation*}

We define stopping times and with them we see the Doob inequalities and the martingales convergence theorems.

One key aspect of the martingales is that one can transform them using an predictable process, a non-anticipating process,
to generate new martingales, this can be helpful to prove equality,
inequality or bounds for non trivial martingales that we encounter like,
combination of martingales or by adding stopping times to our martingales, so we can use the known theorems on these new martingales.

See Probability Theory by Achim Klenke for a deeper and more rigorous treatment of the subject.
