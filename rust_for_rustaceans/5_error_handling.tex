\section{Error Handling}

Functions that can error in rust must return a \textit{Result<T, E>}, where the \textit{E} represents de error that is returned.
Errors can be handled in tow main ways

\subsection{Enumerate Errors}

If the error can have different type and response depending on where it is generated,
it's useful to return an \textit{enum} indicating the kind of error so the called can identify it and decide how to handle it.

In this case the types on the enumeration can can have different traits depending on their characteristics,
but \textbf{all} must implement the \textit{std::error::Error} trait,
and as long as possible the \textit{Display} and \textit{Debug} traits, and if possible the \textit{Sync} and \textit{Send} traits are also important.

\subsection{Opaque Errors}

If the errors on the other hand just provide information that will be useful to the user but not to the program in order to perform different things,
we should use the a single opaque type error that contains the traits \textit{Error, Send, Debug} and \textit{Display}.

\subsection{Propagating Errors}

When a function can throw an error and the caller function don't have to take care of it we can just propagate it back.

Rust's \textit{?} operator acts as a shorthand for \textit{unwrap or return early}.
It helps working easily with errors so they can be propagated back without to verbose.