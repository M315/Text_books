\section{Brownian Motion}

A stochastic process ${B_t}$ is a Brownian motion if

\begin{enumerate}
    \item $B_0 = 0$,
    \item It has independent increments,
    \item For any $0 \leq s \leq t \leq T$ the increment $B_t - B_s$ has a Gaussian distribution with mean 0 and variance $t - s$,
    \item It's continuous almost everywhere.
\end{enumerate}

The definition already gives us a lot of information about the process,
as the increments being Gaussian and independent imply that the distributions of the increments are Gaussian random variables with a diagonalizable covariance matrix.

An other important feature is that it's a Gaussian process,
as the Gaussian characteristic function is determined by only it's first two moments it's quite useful for many problems.

The Covariance of two instances $B_s$ and $B_t$ of a Brownian motion is $Cov(B_s, B_t) = \min(s, t)$.

Finally we construct the Brownian motion by approximating it with wavelet series on $\mathbb{L}^2$,
first by approximating a standard Gaussian process and the by integrating the series, as the Brownian motion is the integral of such process.

This construction is useful to simulate computationally the path of the Brownian motion.