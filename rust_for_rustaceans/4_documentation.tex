\section{Documentation}

The best thing for a project that is big enough or is shared by multiple users is to have a good documentation.

Documentation should explain the obvious things that are in the interface like types, traits, functions…
but more important it should contain all the unexpected behavior like errors…

Also it should contain no obvious examples of the interface,
so the users can start their projects with an example and modifying things until they get the result they were looking for.

\subsection{Type guidance}

A useful thing to do is to create dummy types to use in the function as a boolean instead of a \textit{true} or \textit{false},
so the signatures and the usage of the function is more intuitive.

For example if an input can be a string so the function does something different depending on the string,
would be interesting to have an enum with types so its more natural the options that are available.
