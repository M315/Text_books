\section{Risk}

\subsection{Metrics}

Two of the main metrics to quantify the portfolio risk are \textit{Value at Risk} \textbf{(VaR)}
and \textit{Expected Shortfall}, or \textit{Conditional Value at Risk} \textbf{(CVar)}.
Both are defined as functions depending on a parameter $p$ representing a probability.

Let $Q$ be the daily portfolio pnl and $F(Q)$ the its pdf, then

\begin{equation}
	p = \int_{-\inf}^{VaR(p)} F(Q) dQ,
\end{equation}

\begin{equation}
	CVaR(p) = \frac{1}{p} \int_{-\inf}^{VaR(p)} Q F(Q) dQ,
\end{equation}

this gives us $CVaR(p) < VaR(p) < 0$.

One important feature is that CVaR is convex in the sence that given two portfolios $X$ and $Y$,
then $CVaR(X + Y) \leq CVaR(X) + CVaR(Y)$, wheras VaR is not.

Computing the VaR and the CVaR requieres a hard to measure distribution of portfolio returns.

\subsection{Factor models}

Let $R_s$ be a random daily return, then the portfolio $P_s$ have a pnl $Q = \mathbf{P} \cdot \mathbf{R}$.
The expectation and the variance of the portfolio pnl are:

\begin{aligned}
	E(Q) & = \sum_s f_s P_s, \\
	Var(Q) & = \sum_{s, s'} P_s P_{s'} Cov(R_s, R_{s'}),
\end{aligned}

where $f_s$ is the forcast of the expected return $E(R_s)$.

The main issue of calculating the risk using the volatility,
is that the covariance $C_{s s'} = Cov(R_s, R_{s'})$ is quite complicated to measure historically and even harder to forecast.

The best way to use the covarianvce matrix is by cleaned of noise and/or simplified by shrinkage.

To shring the matrix we  can 
