\section{Risk}

\subsection{Metrics}

Two of the main metrics to quantify the portfolio risk are \textit{Value at Risk} \textbf{(VaR)}
and \textit{Expected Shortfall}, or \textit{Conditional Value at Risk} \textbf{(CVar)}.
Both are defined as functions depending on a parameter $p$ representing a probability.

Let $Q$ be the daily portfolio pnl and $F(Q)$ the its pdf, then

\begin{equation}
	p = \int_{-\inf}^{VaR(p)} F(Q) dQ,
\end{equation}

\begin{equation}
	CVaR(p) = \frac{1}{p} \int_{-\inf}^{VaR(p)} Q F(Q) dQ,
\end{equation}

this gives us $CVaR(p) < VaR(p) < 0$.

One important feature is that CVaR is convex in the sence that given two portfolios $X$ and $Y$,
then $CVaR(X + Y) \leq CVaR(X) + CVaR(Y)$, wheras VaR is not.

Computing the VaR and the CVaR requieres a hard to measure distribution of portfolio returns.
