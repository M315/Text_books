\section{Market Data}

Financial data can be presented by ticks, that are all the movements of a given market,
or by agregated metrics like, open price, close price, volume, minute bars, \dots

The data must be complete and when doing backtesting or simulations we must keep the values that had been droped from the marked.

We need also to adjust the data to changes of an asset like dividents and splits.
Those can be traked backward or forward.
The downside of doing it backwards is that each day all the past adjustments must be recomputed.

The returns are usually presented in linear way (accounting way) or logarithmic way.
The upside of the logarithmic way is that the returns are additive serially, while with the linear way they are not.
The logarithmic returns are a bit off from the linear returns,
this is important when we compute expectations on returns, so we must adjust them using the volatility.

